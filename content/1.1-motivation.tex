Data carving is a forensic process that attempts to recover files without previous information of where the file starts or ends \cite{garfinkel_carving_2007}.
To accomplish this, a software has to analyze a source of raw data, searching for patterns indicating a known file type and making attempts to locate and reconstruct each of its constituent parts.
That process commonly disregards the filesystem \cite{veenman_statistical_2007}, being able to recover deleted files from unallocated areas, but faces the problem of fragmentation \cite{veenman_statistical_2007}  \cite{pal_evolution_2009}: in many cases, files are not written sequentially on disk and deleted files may have missing parts.

For example, while doing data carving on a hard drive, a software could sequentially read each drive sector, find a known header of a video file and save the following sectors until a footer is found or a size limit is reached. This is a common data carving approach, but one that fails to recover fragmented videos.

The patterns searched by data carving software are generally manually coded, taking advantage of fixed byte sequences found on headers and footers. But the amount of different file types combined with the slow process of manually coding each of those patterns makes the development of data carving software a tedious task \cite{mcdaniel_content_2003}.

The application of machine learning solutions to this manual task has the potential to make it easier and faster. The two tasks that could most likely benefit from the usage of machine learning techniques are file fragment classification and file reassembling. File fragment classification aims to identify the original type of file from which a given block of data was extracted. Reassembling is the attempt to reconstruct a file from pieces that may be out of order and mixed with pieces from other files.

Sequence labeling is a branch of machine learning that seeks to attribute labels to a sequence of instances related to each other.

This work describes the current state of art in data carving and in sequence labeling, exploring how neural networks can be applied in file fragment classification and reassembling.