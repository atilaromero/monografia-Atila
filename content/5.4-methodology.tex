\section{Metrics and datasets}

In order to compare solutions, a methodology of comparison must be defined in the early stages of the research, specifying datasets and metrics.

For datasets, some of the most used ones were the Digital Forensic Research Workshop (DFRWS) 2006-2007 dataset \cite{qiu_new_2014}, \cite{ali_review_2018}, and GovDocs \cite{hiester_file_2018}, \cite{fitzgerald_using_2012}, \cite{beebe_sceadan:_2013}.

Classic data carving tools, namely Scalpel \cite{richard_iii_scalpel:_2005}, Foremost \cite{kendall_foremost_2019} and Photorec \cite{grenier_photorec_2019}, can be used to process those datasets, to create a research baseline.

For metrics, usual choices like accuracy, precision, recall, and f1-score can be used to measure the quality of the results. Model training time is also an important metric to consider.

To measure the ease with which the end user could add new file types, a possible solution is to use a survey, but that can only be made in the later stages of the work after a minimal working tool is devised.


\section{Testing environment}
Before starting the tests, an appropriate environment must be built. This requires an evaluation of available machine learning tools and frameworks. Prominent options include Keras \cite{chollet_keras_2019} and TensorFlow \cite{google_brain_tensorflow_2019}.

During evaluation of those tools, while building the environment, some of the most simple solution alternatives can begin to be tested, probably training binary classification of file types, using a simple dataset consisting of small files. This will allow to both compare frameworks and to begin the tests.

\section{Experiments}
Following the environment preparation, the next step should be to compare solutions, starting with the most simple solutions first, and increasing complexity next. Planned neural networks to test include feedforward, convolutional, LSTM and BLSTM. For comparison of results, SVM and kNN can also be explored.

Some of the possible experiments that can be conducted are: 
%
manipulation of datasets to measure results on different scenarios,  shuffling data to simulate fragmentation, for example, or by removing portions of files to simulate data corruption;
%
increase in the number of supported file types, investigating the best strategy to scale the solution;
%
try to reassemble fragmented files using different neural network architectures;
%
research ways to share trained models, an essential requirement to give practical applicability to the research;
%. Some alternatives should be probed and some solutions outlined.
%
adaption of visualization techniques of neural networks, attempting to infer file structure.
