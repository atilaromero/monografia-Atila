The data carving field, despite being actively studied, does not have the number of research papers that other fields have. This implies that improvements in current data carving techniques are likely to use knowledge transferred from other areas. While in a machine learning search the excess of relevant papers becomes a problem, an equivalent data carving search may face a scarcity of relevant papers.

% \todo[inline]{review this}
It is important to notice that different classes of algorithms would be suitable for different learning objectives.
A clustering algorithm could be used to group fragments of data with unknown structure.
A classification algorithm could be used to attribute labels to each disk sector, which is relevant to file fragment classification. 
Sequence labeling considers relations between labels on a sequence, which is often the case when classifying consecutive sectors of a disk. It has potential to be used in file fragment classification and in reassembling of fragmented data.

Sequence labeling deals with the task of attributing categorical labels to a group of instances where, unlike happens with other classification problems, those instance labels are not independent.
One example of such task is known as Part Of Speech (POS) tagging, where the goal is, given a word in a sentence, classify it, for example as a noun or a verb.
The classification of each word is not independent of its surroundings since it provides context from which the meaning of the word may be inferred.

Another example of a sequence labeling task is found in speech recognition.
The sound of a spoken sentence is split into parts and each part receives a label corresponding to a phoneme. But those labels are not independent since some combinations of phonemes are meaningful while others are not. Thus, taking this context into account increases the accuracy of the results.

From those classes of pattern recognition approaches mentioned above, both classification algorithms and sequence labeling algorithms are potentially good choices.
