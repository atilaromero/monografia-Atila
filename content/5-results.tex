%\levelB{Dataset}
This study uses the Govdocs1 dataset \cite{garfinkel_bringing_2009}, which was fully downloaded and then organized by file extension. This dataset has files with 63 different extensions. The 33 extensions with less than 200 files were discarded. The  ``text'' and ``unk'' extensions were discarded because examination showed that files with these extensions use multiple formats and they do not correspond to a single file type. From the remaining 28 extensions, listed in table \ref{tab:govdocs1}, 200 files of each were randomly selected, 100 to use in the training dataset and 100 to use in the validation dataset.

\begin{table}[!ht]
    \centering
    \caption{Govdocs1 dataset}
    \label{tab:govdocs1}
\begin{tabular}{|l|l|l|}
\hline
Extension & Number of files & Number of blocks \\ \hline
                                                  \hline
pdf       & 231232          & 268071071    \\ \hline
html      & 214568          & 25710908     \\ \hline
jpg       & 109233          & 73242253     \\ \hline
txt       & 78286           & 99435540     \\ \hline
doc       & 76616           & 60654930     \\ \hline
xls       & 62635           & 58718224     \\ \hline
ppt       & 49702           & 251210471    \\ \hline
gif       & 36302           & 5962516      \\ \hline
xml       & 33458           & 16954875     \\ \hline
ps        & 22015           & 56547464     \\ \hline
csv       & 18360           & 6843009      \\ \hline
gz        & 13725           & 17748905     \\ \hline
log       & 9976            & 8467819      \\ \hline
eps       & 5191            & 5756138      \\ \hline
% unk       & 5186            & 2983922      \\ \hline
png       & 4125            & 2207489      \\ \hline
swf       & 3476            & 3798321      \\ \hline
dbase3    & 2601            & 38972        \\ \hline
pps       & 1619            & 7432480      \\ \hline
rtf       & 1125            & 958239       \\ \hline
kml       & 993             & 309422       \\ \hline
kmz       & 943             & 549462       \\ \hline
% text      & 839             & 1527118      \\ \hline
hlp       & 659             & 8692         \\ \hline
f         & 602             & 94543        \\ \hline
sql       & 462             & 244634       \\ \hline
wp        & 364             & 87643        \\ \hline
dwf       & 299             & 85500        \\ \hline
java      & 292             & 14530        \\ \hline
pptx      & 215             & 1151796      \\ \hline
% fits      & 182             & 678128       \\ \hline
% tmp       & 180             & 28426        \\ \hline
% tex       & 163             & 10520        \\ \hline
% docx      & 163             & 65969        \\ \hline
% troff     & 110             & 8020         \\ \hline
% bmp       & 72              & 62686        \\ \hline
% sgml      & 62              & 44138        \\ \hline
% gls       & 60              & 517          \\ \hline
% pub       & 55              & 1421         \\ \hline
% xlsx      & 37              & 12910        \\ \hline
% fm        & 25              & 6717         \\ \hline
% zip       & 10              & 31525        \\ \hline
% ttf       & 10              & 1540         \\ \hline
% xbm       & 8               & 578          \\ \hline
% wk1       & 7               & 6493         \\ \hline
% sys       & 7               & 15           \\ \hline
% ileaf     & 4               & 1656         \\ \hline
% exported  & 3               & 324          \\ \hline
% data      & 3               & 1733         \\ \hline
% odp       & 2               & 2384         \\ \hline
% mac       & 2               & 0            \\ \hline
% lnk       & 2               & 2            \\ \hline
% js        & 2               & 36           \\ \hline
% g3        & 2               & 498          \\ \hline
% chp       & 2               & 73           \\ \hline
% 123       & 2               & 434          \\ \hline
% wk3       & 1               & 229          \\ \hline
% vrml      & 1               & 660          \\ \hline
% squeak    & 1               & 25354        \\ \hline
% py        & 1               & 480          \\ \hline
% pst       & 1               & 20           \\ \hline
% icns      & 1               & 0            \\ \hline
% bin       & 1               & 7            \\ \hline
\end{tabular}
\end{table}


%\levelB{Hardware}
The experiments did not take advantage of GPU acceleration and were  conducted on a single computer with 256GB of RAM and with 2 Intel\textregistered Xeon\textregistered E5-2630 v2 processors, with 6 cores each, with 2 hyper-threads per core, or 24 hyper-threads in total. 


%\levelB{Software}
The main software and frameworks used to build the experiments were Python 3.6, Jupyter notebook, Tensorflow 1.14.0, Keras 2.2.4-tf, and Fedora Linux 27.
% repository
The source code for the experiments is available at \sloppy\url{http://github.com/atilaromero/carving-experiments}.

%\levelB{Sampling}
To select a sample instance from the Govdocs1 dataset, first a random file is selected among those available, without replacement. Then, a block from this file is randomly selected. After all files have participated in the process, the process may be repeated as long as necessary. This way, all files are considered and the classes are easily balanced.
This contrasts with the sampling technique applied in other works, where all the sectors are grouped before sampling, which may lead to an imbalance between classes.
\noindent
\begin{figure*}[htb!]
\centering\includegraphics[width=0.9\textwidth]{content/epoch_val_categorical_accuracy.png}
\caption{\label{fig:learning}Learning curves of some models (validation accuracy vs. time)}%
\end{figure*}

%\levelB{Inputs}
%one-hot
The input features of the network for each instance is a 512x256 matrix representing only one block of 512 bytes of a random file of the dataset. Each of the 512 bytes is one-hot encoded, meaning that its value is converted into a vector with 256 elements, with only one of them set to 1, corresponding to the value of the byte, while the others are set to zero. A batch size of 100 was used, with 28 steps per epoch.

%8bits
% The input features of the network for each instance is a 512x8 matrix representing only one block of 512 bytes of a random file of the dataset. Each of the 512 bytes uses a custom encoding where each of the 8 bits of a byte is represented as -1 or 1, depending whether the bit is 0 or 1. During initial tests, this encoding was compared to three other encodings: one-hot encoding, 8 bits represented as 0 or 1\todo{include citation}, and 8 bits represented as [0,1] and [1,0] \cite{hiester_file_2018}. More research should be done in the future to determine the best of the four, but initial results suggest they have similar impact on the model accuracy. The one-hot encoding has the disadvantage of increasing the input matrix size by a factor of 32.

%\levelB{Outputs}
The output of the network for a given instance is a vector with a size equal to the number of classes, subjected to a softmax function, which applies the exponential function on the vector and then normalizes it. Each value will represent the predicted probability that the instance belongs to a specific class.

%exp18
The two models that used LSTM layers without convolutional layers presented slower learning in comparison with the others, as can be seen in figure \ref{fig:learning}. The validation accuracy values can be consulted in figure \ref{fig:models}.


\noindent
\begin{figure*}[htb!]
\centering\includegraphics[width=1.0\textwidth]{content/models.png}
\caption{\label{fig:models}The bar plot shows the validation accuracy of the considered models. Their training time, in minutes, and the total epochs processed is also indicated. The training was interrupted when no further improvement was observed for 10 consecutive epochs.}%
\end{figure*}


The validation accuracy values of the remaining models were in the 0.35-0.54 range.

% The selected model for further research, identified as ``CM'', achieved an accuracy of 0.54. The models, ``CLD'' and ``CCM'' achieved similar values, but model ``CM'' performed better in the experiment described in section \ref{sec:exprandom}.

To check if the two slower models could give better results if executed for a longer period, they were trained for 10 hours. The model identified as ``LD'', which uses an LSTM layer followed by a fully connected layer, was able to achieve an accuracy of 0.494, while the other, ``LL'', achieved only 0.153.

%stagnation in learning with such short times, combined with low final accuracy suggests that the dataset present some patterns that are easily recognizable, while the rest of the dataset present a very hard classification task.

% \levelC{Limitations and threats to validity}

The model CLD was chosen to be compared with other works because it was among the 3 models with highest accuracy, but achieved its results in less time and showed high resilience to changes in parameters during the tests, while the others did not.
 
In this second stage, using longer training sessions, the accuracy for the 28 file types used in the first stage increased from 0.54 to 0.62. Using the same file types of other works, the CLD model achieved an accuracy of 
0.66 for the file types from Chen \textit{et al.} \cite{chen_file_2018},
0.91 for Hiester \cite{hiester_file_2018}, 
0.59 for Wang \textit{et al.} \cite{wang_sparse_2018},
0.64 for Wang \textit{et al.} \cite{wang_file_2018},
and
0.60 for Vulinović \textit{et al.} \cite{vulinovic_neural_2019}.
The results can be seen in figure \ref{fig:cldothers}.

\noindent
\begin{figure*}[htb!]
\centering\includegraphics[width=0.8\textwidth]{content/CLD-others.png}
\caption[CLD vs. other studies]{\label{fig:cldothers}The CLD model was trained from scratch using the same file types used in other works. The bar plot shows the comparison between the validation accuracy obtained with CLD and those achieved in other studies.}%
\end{figure*}